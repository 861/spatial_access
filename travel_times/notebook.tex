
% Default to the notebook output style

    


% Inherit from the specified cell style.




    
\documentclass[11pt]{article}

    
    
    \usepackage[T1]{fontenc}
    % Nicer default font (+ math font) than Computer Modern for most use cases
    \usepackage{mathpazo}

    % Basic figure setup, for now with no caption control since it's done
    % automatically by Pandoc (which extracts ![](path) syntax from Markdown).
    \usepackage{graphicx}
    % We will generate all images so they have a width \maxwidth. This means
    % that they will get their normal width if they fit onto the page, but
    % are scaled down if they would overflow the margins.
    \makeatletter
    \def\maxwidth{\ifdim\Gin@nat@width>\linewidth\linewidth
    \else\Gin@nat@width\fi}
    \makeatother
    \let\Oldincludegraphics\includegraphics
    % Set max figure width to be 80% of text width, for now hardcoded.
    \renewcommand{\includegraphics}[1]{\Oldincludegraphics[width=.8\maxwidth]{#1}}
    % Ensure that by default, figures have no caption (until we provide a
    % proper Figure object with a Caption API and a way to capture that
    % in the conversion process - todo).
    \usepackage{caption}
    \DeclareCaptionLabelFormat{nolabel}{}
    \captionsetup{labelformat=nolabel}

    \usepackage{adjustbox} % Used to constrain images to a maximum size 
    \usepackage{xcolor} % Allow colors to be defined
    \usepackage{enumerate} % Needed for markdown enumerations to work
    \usepackage{geometry} % Used to adjust the document margins
    \usepackage{amsmath} % Equations
    \usepackage{amssymb} % Equations
    \usepackage{textcomp} % defines textquotesingle
    % Hack from http://tex.stackexchange.com/a/47451/13684:
    \AtBeginDocument{%
        \def\PYZsq{\textquotesingle}% Upright quotes in Pygmentized code
    }
    \usepackage{upquote} % Upright quotes for verbatim code
    \usepackage{eurosym} % defines \euro
    \usepackage[mathletters]{ucs} % Extended unicode (utf-8) support
    \usepackage[utf8x]{inputenc} % Allow utf-8 characters in the tex document
    \usepackage{fancyvrb} % verbatim replacement that allows latex
    \usepackage{grffile} % extends the file name processing of package graphics 
                         % to support a larger range 
    % The hyperref package gives us a pdf with properly built
    % internal navigation ('pdf bookmarks' for the table of contents,
    % internal cross-reference links, web links for URLs, etc.)
    \usepackage{hyperref}
    \usepackage{longtable} % longtable support required by pandoc >1.10
    \usepackage{booktabs}  % table support for pandoc > 1.12.2
    \usepackage[inline]{enumitem} % IRkernel/repr support (it uses the enumerate* environment)
    \usepackage[normalem]{ulem} % ulem is needed to support strikethroughs (\sout)
                                % normalem makes italics be italics, not underlines
    

    
    
    % Colors for the hyperref package
    \definecolor{urlcolor}{rgb}{0,.145,.698}
    \definecolor{linkcolor}{rgb}{.71,0.21,0.01}
    \definecolor{citecolor}{rgb}{.12,.54,.11}

    % ANSI colors
    \definecolor{ansi-black}{HTML}{3E424D}
    \definecolor{ansi-black-intense}{HTML}{282C36}
    \definecolor{ansi-red}{HTML}{E75C58}
    \definecolor{ansi-red-intense}{HTML}{B22B31}
    \definecolor{ansi-green}{HTML}{00A250}
    \definecolor{ansi-green-intense}{HTML}{007427}
    \definecolor{ansi-yellow}{HTML}{DDB62B}
    \definecolor{ansi-yellow-intense}{HTML}{B27D12}
    \definecolor{ansi-blue}{HTML}{208FFB}
    \definecolor{ansi-blue-intense}{HTML}{0065CA}
    \definecolor{ansi-magenta}{HTML}{D160C4}
    \definecolor{ansi-magenta-intense}{HTML}{A03196}
    \definecolor{ansi-cyan}{HTML}{60C6C8}
    \definecolor{ansi-cyan-intense}{HTML}{258F8F}
    \definecolor{ansi-white}{HTML}{C5C1B4}
    \definecolor{ansi-white-intense}{HTML}{A1A6B2}

    % commands and environments needed by pandoc snippets
    % extracted from the output of `pandoc -s`
    \providecommand{\tightlist}{%
      \setlength{\itemsep}{0pt}\setlength{\parskip}{0pt}}
    \DefineVerbatimEnvironment{Highlighting}{Verbatim}{commandchars=\\\{\}}
    % Add ',fontsize=\small' for more characters per line
    \newenvironment{Shaded}{}{}
    \newcommand{\KeywordTok}[1]{\textcolor[rgb]{0.00,0.44,0.13}{\textbf{{#1}}}}
    \newcommand{\DataTypeTok}[1]{\textcolor[rgb]{0.56,0.13,0.00}{{#1}}}
    \newcommand{\DecValTok}[1]{\textcolor[rgb]{0.25,0.63,0.44}{{#1}}}
    \newcommand{\BaseNTok}[1]{\textcolor[rgb]{0.25,0.63,0.44}{{#1}}}
    \newcommand{\FloatTok}[1]{\textcolor[rgb]{0.25,0.63,0.44}{{#1}}}
    \newcommand{\CharTok}[1]{\textcolor[rgb]{0.25,0.44,0.63}{{#1}}}
    \newcommand{\StringTok}[1]{\textcolor[rgb]{0.25,0.44,0.63}{{#1}}}
    \newcommand{\CommentTok}[1]{\textcolor[rgb]{0.38,0.63,0.69}{\textit{{#1}}}}
    \newcommand{\OtherTok}[1]{\textcolor[rgb]{0.00,0.44,0.13}{{#1}}}
    \newcommand{\AlertTok}[1]{\textcolor[rgb]{1.00,0.00,0.00}{\textbf{{#1}}}}
    \newcommand{\FunctionTok}[1]{\textcolor[rgb]{0.02,0.16,0.49}{{#1}}}
    \newcommand{\RegionMarkerTok}[1]{{#1}}
    \newcommand{\ErrorTok}[1]{\textcolor[rgb]{1.00,0.00,0.00}{\textbf{{#1}}}}
    \newcommand{\NormalTok}[1]{{#1}}
    
    % Additional commands for more recent versions of Pandoc
    \newcommand{\ConstantTok}[1]{\textcolor[rgb]{0.53,0.00,0.00}{{#1}}}
    \newcommand{\SpecialCharTok}[1]{\textcolor[rgb]{0.25,0.44,0.63}{{#1}}}
    \newcommand{\VerbatimStringTok}[1]{\textcolor[rgb]{0.25,0.44,0.63}{{#1}}}
    \newcommand{\SpecialStringTok}[1]{\textcolor[rgb]{0.73,0.40,0.53}{{#1}}}
    \newcommand{\ImportTok}[1]{{#1}}
    \newcommand{\DocumentationTok}[1]{\textcolor[rgb]{0.73,0.13,0.13}{\textit{{#1}}}}
    \newcommand{\AnnotationTok}[1]{\textcolor[rgb]{0.38,0.63,0.69}{\textbf{\textit{{#1}}}}}
    \newcommand{\CommentVarTok}[1]{\textcolor[rgb]{0.38,0.63,0.69}{\textbf{\textit{{#1}}}}}
    \newcommand{\VariableTok}[1]{\textcolor[rgb]{0.10,0.09,0.49}{{#1}}}
    \newcommand{\ControlFlowTok}[1]{\textcolor[rgb]{0.00,0.44,0.13}{\textbf{{#1}}}}
    \newcommand{\OperatorTok}[1]{\textcolor[rgb]{0.40,0.40,0.40}{{#1}}}
    \newcommand{\BuiltInTok}[1]{{#1}}
    \newcommand{\ExtensionTok}[1]{{#1}}
    \newcommand{\PreprocessorTok}[1]{\textcolor[rgb]{0.74,0.48,0.00}{{#1}}}
    \newcommand{\AttributeTok}[1]{\textcolor[rgb]{0.49,0.56,0.16}{{#1}}}
    \newcommand{\InformationTok}[1]{\textcolor[rgb]{0.38,0.63,0.69}{\textbf{\textit{{#1}}}}}
    \newcommand{\WarningTok}[1]{\textcolor[rgb]{0.38,0.63,0.69}{\textbf{\textit{{#1}}}}}
    
    
    % Define a nice break command that doesn't care if a line doesn't already
    % exist.
    \def\br{\hspace*{\fill} \\* }
    % Math Jax compatability definitions
    \def\gt{>}
    \def\lt{<}
    % Document parameters
    \title{1. matrix}
    
    
    

    % Pygments definitions
    
\makeatletter
\def\PY@reset{\let\PY@it=\relax \let\PY@bf=\relax%
    \let\PY@ul=\relax \let\PY@tc=\relax%
    \let\PY@bc=\relax \let\PY@ff=\relax}
\def\PY@tok#1{\csname PY@tok@#1\endcsname}
\def\PY@toks#1+{\ifx\relax#1\empty\else%
    \PY@tok{#1}\expandafter\PY@toks\fi}
\def\PY@do#1{\PY@bc{\PY@tc{\PY@ul{%
    \PY@it{\PY@bf{\PY@ff{#1}}}}}}}
\def\PY#1#2{\PY@reset\PY@toks#1+\relax+\PY@do{#2}}

\expandafter\def\csname PY@tok@w\endcsname{\def\PY@tc##1{\textcolor[rgb]{0.73,0.73,0.73}{##1}}}
\expandafter\def\csname PY@tok@c\endcsname{\let\PY@it=\textit\def\PY@tc##1{\textcolor[rgb]{0.25,0.50,0.50}{##1}}}
\expandafter\def\csname PY@tok@cp\endcsname{\def\PY@tc##1{\textcolor[rgb]{0.74,0.48,0.00}{##1}}}
\expandafter\def\csname PY@tok@k\endcsname{\let\PY@bf=\textbf\def\PY@tc##1{\textcolor[rgb]{0.00,0.50,0.00}{##1}}}
\expandafter\def\csname PY@tok@kp\endcsname{\def\PY@tc##1{\textcolor[rgb]{0.00,0.50,0.00}{##1}}}
\expandafter\def\csname PY@tok@kt\endcsname{\def\PY@tc##1{\textcolor[rgb]{0.69,0.00,0.25}{##1}}}
\expandafter\def\csname PY@tok@o\endcsname{\def\PY@tc##1{\textcolor[rgb]{0.40,0.40,0.40}{##1}}}
\expandafter\def\csname PY@tok@ow\endcsname{\let\PY@bf=\textbf\def\PY@tc##1{\textcolor[rgb]{0.67,0.13,1.00}{##1}}}
\expandafter\def\csname PY@tok@nb\endcsname{\def\PY@tc##1{\textcolor[rgb]{0.00,0.50,0.00}{##1}}}
\expandafter\def\csname PY@tok@nf\endcsname{\def\PY@tc##1{\textcolor[rgb]{0.00,0.00,1.00}{##1}}}
\expandafter\def\csname PY@tok@nc\endcsname{\let\PY@bf=\textbf\def\PY@tc##1{\textcolor[rgb]{0.00,0.00,1.00}{##1}}}
\expandafter\def\csname PY@tok@nn\endcsname{\let\PY@bf=\textbf\def\PY@tc##1{\textcolor[rgb]{0.00,0.00,1.00}{##1}}}
\expandafter\def\csname PY@tok@ne\endcsname{\let\PY@bf=\textbf\def\PY@tc##1{\textcolor[rgb]{0.82,0.25,0.23}{##1}}}
\expandafter\def\csname PY@tok@nv\endcsname{\def\PY@tc##1{\textcolor[rgb]{0.10,0.09,0.49}{##1}}}
\expandafter\def\csname PY@tok@no\endcsname{\def\PY@tc##1{\textcolor[rgb]{0.53,0.00,0.00}{##1}}}
\expandafter\def\csname PY@tok@nl\endcsname{\def\PY@tc##1{\textcolor[rgb]{0.63,0.63,0.00}{##1}}}
\expandafter\def\csname PY@tok@ni\endcsname{\let\PY@bf=\textbf\def\PY@tc##1{\textcolor[rgb]{0.60,0.60,0.60}{##1}}}
\expandafter\def\csname PY@tok@na\endcsname{\def\PY@tc##1{\textcolor[rgb]{0.49,0.56,0.16}{##1}}}
\expandafter\def\csname PY@tok@nt\endcsname{\let\PY@bf=\textbf\def\PY@tc##1{\textcolor[rgb]{0.00,0.50,0.00}{##1}}}
\expandafter\def\csname PY@tok@nd\endcsname{\def\PY@tc##1{\textcolor[rgb]{0.67,0.13,1.00}{##1}}}
\expandafter\def\csname PY@tok@s\endcsname{\def\PY@tc##1{\textcolor[rgb]{0.73,0.13,0.13}{##1}}}
\expandafter\def\csname PY@tok@sd\endcsname{\let\PY@it=\textit\def\PY@tc##1{\textcolor[rgb]{0.73,0.13,0.13}{##1}}}
\expandafter\def\csname PY@tok@si\endcsname{\let\PY@bf=\textbf\def\PY@tc##1{\textcolor[rgb]{0.73,0.40,0.53}{##1}}}
\expandafter\def\csname PY@tok@se\endcsname{\let\PY@bf=\textbf\def\PY@tc##1{\textcolor[rgb]{0.73,0.40,0.13}{##1}}}
\expandafter\def\csname PY@tok@sr\endcsname{\def\PY@tc##1{\textcolor[rgb]{0.73,0.40,0.53}{##1}}}
\expandafter\def\csname PY@tok@ss\endcsname{\def\PY@tc##1{\textcolor[rgb]{0.10,0.09,0.49}{##1}}}
\expandafter\def\csname PY@tok@sx\endcsname{\def\PY@tc##1{\textcolor[rgb]{0.00,0.50,0.00}{##1}}}
\expandafter\def\csname PY@tok@m\endcsname{\def\PY@tc##1{\textcolor[rgb]{0.40,0.40,0.40}{##1}}}
\expandafter\def\csname PY@tok@gh\endcsname{\let\PY@bf=\textbf\def\PY@tc##1{\textcolor[rgb]{0.00,0.00,0.50}{##1}}}
\expandafter\def\csname PY@tok@gu\endcsname{\let\PY@bf=\textbf\def\PY@tc##1{\textcolor[rgb]{0.50,0.00,0.50}{##1}}}
\expandafter\def\csname PY@tok@gd\endcsname{\def\PY@tc##1{\textcolor[rgb]{0.63,0.00,0.00}{##1}}}
\expandafter\def\csname PY@tok@gi\endcsname{\def\PY@tc##1{\textcolor[rgb]{0.00,0.63,0.00}{##1}}}
\expandafter\def\csname PY@tok@gr\endcsname{\def\PY@tc##1{\textcolor[rgb]{1.00,0.00,0.00}{##1}}}
\expandafter\def\csname PY@tok@ge\endcsname{\let\PY@it=\textit}
\expandafter\def\csname PY@tok@gs\endcsname{\let\PY@bf=\textbf}
\expandafter\def\csname PY@tok@gp\endcsname{\let\PY@bf=\textbf\def\PY@tc##1{\textcolor[rgb]{0.00,0.00,0.50}{##1}}}
\expandafter\def\csname PY@tok@go\endcsname{\def\PY@tc##1{\textcolor[rgb]{0.53,0.53,0.53}{##1}}}
\expandafter\def\csname PY@tok@gt\endcsname{\def\PY@tc##1{\textcolor[rgb]{0.00,0.27,0.87}{##1}}}
\expandafter\def\csname PY@tok@err\endcsname{\def\PY@bc##1{\setlength{\fboxsep}{0pt}\fcolorbox[rgb]{1.00,0.00,0.00}{1,1,1}{\strut ##1}}}
\expandafter\def\csname PY@tok@kc\endcsname{\let\PY@bf=\textbf\def\PY@tc##1{\textcolor[rgb]{0.00,0.50,0.00}{##1}}}
\expandafter\def\csname PY@tok@kd\endcsname{\let\PY@bf=\textbf\def\PY@tc##1{\textcolor[rgb]{0.00,0.50,0.00}{##1}}}
\expandafter\def\csname PY@tok@kn\endcsname{\let\PY@bf=\textbf\def\PY@tc##1{\textcolor[rgb]{0.00,0.50,0.00}{##1}}}
\expandafter\def\csname PY@tok@kr\endcsname{\let\PY@bf=\textbf\def\PY@tc##1{\textcolor[rgb]{0.00,0.50,0.00}{##1}}}
\expandafter\def\csname PY@tok@bp\endcsname{\def\PY@tc##1{\textcolor[rgb]{0.00,0.50,0.00}{##1}}}
\expandafter\def\csname PY@tok@fm\endcsname{\def\PY@tc##1{\textcolor[rgb]{0.00,0.00,1.00}{##1}}}
\expandafter\def\csname PY@tok@vc\endcsname{\def\PY@tc##1{\textcolor[rgb]{0.10,0.09,0.49}{##1}}}
\expandafter\def\csname PY@tok@vg\endcsname{\def\PY@tc##1{\textcolor[rgb]{0.10,0.09,0.49}{##1}}}
\expandafter\def\csname PY@tok@vi\endcsname{\def\PY@tc##1{\textcolor[rgb]{0.10,0.09,0.49}{##1}}}
\expandafter\def\csname PY@tok@vm\endcsname{\def\PY@tc##1{\textcolor[rgb]{0.10,0.09,0.49}{##1}}}
\expandafter\def\csname PY@tok@sa\endcsname{\def\PY@tc##1{\textcolor[rgb]{0.73,0.13,0.13}{##1}}}
\expandafter\def\csname PY@tok@sb\endcsname{\def\PY@tc##1{\textcolor[rgb]{0.73,0.13,0.13}{##1}}}
\expandafter\def\csname PY@tok@sc\endcsname{\def\PY@tc##1{\textcolor[rgb]{0.73,0.13,0.13}{##1}}}
\expandafter\def\csname PY@tok@dl\endcsname{\def\PY@tc##1{\textcolor[rgb]{0.73,0.13,0.13}{##1}}}
\expandafter\def\csname PY@tok@s2\endcsname{\def\PY@tc##1{\textcolor[rgb]{0.73,0.13,0.13}{##1}}}
\expandafter\def\csname PY@tok@sh\endcsname{\def\PY@tc##1{\textcolor[rgb]{0.73,0.13,0.13}{##1}}}
\expandafter\def\csname PY@tok@s1\endcsname{\def\PY@tc##1{\textcolor[rgb]{0.73,0.13,0.13}{##1}}}
\expandafter\def\csname PY@tok@mb\endcsname{\def\PY@tc##1{\textcolor[rgb]{0.40,0.40,0.40}{##1}}}
\expandafter\def\csname PY@tok@mf\endcsname{\def\PY@tc##1{\textcolor[rgb]{0.40,0.40,0.40}{##1}}}
\expandafter\def\csname PY@tok@mh\endcsname{\def\PY@tc##1{\textcolor[rgb]{0.40,0.40,0.40}{##1}}}
\expandafter\def\csname PY@tok@mi\endcsname{\def\PY@tc##1{\textcolor[rgb]{0.40,0.40,0.40}{##1}}}
\expandafter\def\csname PY@tok@il\endcsname{\def\PY@tc##1{\textcolor[rgb]{0.40,0.40,0.40}{##1}}}
\expandafter\def\csname PY@tok@mo\endcsname{\def\PY@tc##1{\textcolor[rgb]{0.40,0.40,0.40}{##1}}}
\expandafter\def\csname PY@tok@ch\endcsname{\let\PY@it=\textit\def\PY@tc##1{\textcolor[rgb]{0.25,0.50,0.50}{##1}}}
\expandafter\def\csname PY@tok@cm\endcsname{\let\PY@it=\textit\def\PY@tc##1{\textcolor[rgb]{0.25,0.50,0.50}{##1}}}
\expandafter\def\csname PY@tok@cpf\endcsname{\let\PY@it=\textit\def\PY@tc##1{\textcolor[rgb]{0.25,0.50,0.50}{##1}}}
\expandafter\def\csname PY@tok@c1\endcsname{\let\PY@it=\textit\def\PY@tc##1{\textcolor[rgb]{0.25,0.50,0.50}{##1}}}
\expandafter\def\csname PY@tok@cs\endcsname{\let\PY@it=\textit\def\PY@tc##1{\textcolor[rgb]{0.25,0.50,0.50}{##1}}}

\def\PYZbs{\char`\\}
\def\PYZus{\char`\_}
\def\PYZob{\char`\{}
\def\PYZcb{\char`\}}
\def\PYZca{\char`\^}
\def\PYZam{\char`\&}
\def\PYZlt{\char`\<}
\def\PYZgt{\char`\>}
\def\PYZsh{\char`\#}
\def\PYZpc{\char`\%}
\def\PYZdl{\char`\$}
\def\PYZhy{\char`\-}
\def\PYZsq{\char`\'}
\def\PYZdq{\char`\"}
\def\PYZti{\char`\~}
% for compatibility with earlier versions
\def\PYZat{@}
\def\PYZlb{[}
\def\PYZrb{]}
\makeatother


    % Exact colors from NB
    \definecolor{incolor}{rgb}{0.0, 0.0, 0.5}
    \definecolor{outcolor}{rgb}{0.545, 0.0, 0.0}



    
    % Prevent overflowing lines due to hard-to-break entities
    \sloppy 
    % Setup hyperref package
    \hypersetup{
      breaklinks=true,  % so long urls are correctly broken across lines
      colorlinks=true,
      urlcolor=urlcolor,
      linkcolor=linkcolor,
      citecolor=citecolor,
      }
    % Slightly bigger margins than the latex defaults
    
    \geometry{verbose,tmargin=1in,bmargin=1in,lmargin=1in,rmargin=1in}
    
    

    \begin{document}
    
    
    \maketitle
    
    

    
    \hypertarget{transit-matrix}{%
\section{Transit Matrix}\label{transit-matrix}}

    \begin{Verbatim}[commandchars=\\\{\}]
{\color{incolor}In [{\color{incolor}3}]:} \PY{k+kn}{import} \PY{n+nn}{sys}\PY{o}{,} \PY{n+nn}{os}
        \PY{n}{os}\PY{o}{.}\PY{n}{chdir}\PY{p}{(}\PY{l+s+s1}{\PYZsq{}}\PY{l+s+s1}{contracts/analytics}\PY{l+s+s1}{\PYZsq{}}\PY{p}{)}
        
        \PY{k+kn}{from} \PY{n+nn}{p2p} \PY{k}{import} \PY{o}{*}
\end{Verbatim}


    \begin{center}\rule{0.5\linewidth}{\linethickness}\end{center}

DEMO

    \textbf{View structure of data example: Health Facilities in Chicago.}\\
Health Facilities Data:
http://makosak.github.io/chihealthaccess/index.html

    \begin{Verbatim}[commandchars=\\\{\}]
{\color{incolor}In [{\color{incolor}97}]:} \PY{n}{df} \PY{o}{=} \PY{n}{pd}\PY{o}{.}\PY{n}{read\PYZus{}csv}\PY{p}{(}\PY{l+s+s1}{\PYZsq{}}\PY{l+s+s1}{resources/marynia\PYZus{}health.csv}\PY{l+s+s1}{\PYZsq{}}\PY{p}{)}
         \PY{n}{df}\PY{o}{.}\PY{n}{head}\PY{p}{(}\PY{p}{)}
\end{Verbatim}


\begin{Verbatim}[commandchars=\\\{\}]
{\color{outcolor}Out[{\color{outcolor}97}]:}    agency\_id                                           facility        lat  \textbackslash{}
         0          1    American Indian Health Service of Chicago, Inc.  41.956676   
         1          2       Hamdard Center for Health and Human Services  41.997852   
         2          3                  Infant Welfare Society of Chicago  41.924904   
         3          4  Mercy Family - Henry Booth House Family Health{\ldots}  41.841694   
         4          6       Cook County - Dr. Jorge Prieto Health Center  41.847143   
         
                  lon  cat\_num  target                category  
         0 -87.651879        5    1000  Other Health Providers  
         1 -87.669535        5    2000  Other Health Providers  
         2 -87.717270        5    1000  Other Health Providers  
         3 -87.624790        5    3000  Other Health Providers  
         4 -87.724975        5    1500  Other Health Providers  
\end{Verbatim}
            
    \hypertarget{distance-matrices}{%
\subsubsection{Distance Matrices}\label{distance-matrices}}

 \textbf{Specifications for the symmetric and asymmetric distance
matrices:}

\begin{itemize}
\tightlist
\item
  network\_type (drive or walk)
\item
  epsilon=0.05 (can change default)\\
\item
  primary\_input\\
\item
  secondary\_input\\
\item
  output\_type=`csv'\\
\item
  n\_best\_matches=4 (for simulations)
\item
  read\_from\_file=None\\
\item
  write\_to\_file (set as True if user wants to save results)\\
\item
  load\_to\_mem=True (True is default but can set it to False if the
  user is running a computational intensive process
  \textgreater{}\textgreater{}\textgreater{}.)
\end{itemize}

\textbf{Please make sure latitude and longitude are correct if using X
and Y.}

\hypertarget{model-1-symmetric-matrix}{%
\subsection{\#\#\# Model 1: Symmetric
Matrix}\label{model-1-symmetric-matrix}}

The first model creates a symmetric distance travel matrix from block to
block (46,357 x 46,367 matrix). Then, we snap the destination points to
the area of analysis (blocks), getting a matrix that calculates the
distance between the destinations and every block in the dataset. ****
There are 46,357 blocks in the city of Chicago, but are reduced to
46,265 so it matches to the community level aggregation.\\
Nevertheless, we need to obtain the distance matrix first (takes approx.
15 min):

    \begin{Verbatim}[commandchars=\\\{\}]
{\color{incolor}In [{\color{incolor} }]:} \PY{c+c1}{\PYZsh{} Specify walking distance matrix (takes \PYZti{}30 min to run) }
        \PY{n}{w\PYZus{}sym\PYZus{}mat} \PY{o}{=} \PY{n}{TransitMatrix}\PY{p}{(}\PY{n}{network\PYZus{}type}\PY{o}{=}\PY{l+s+s1}{\PYZsq{}}\PY{l+s+s1}{walk}\PY{l+s+s1}{\PYZsq{}}\PY{p}{,}
                                  \PY{n}{primary\PYZus{}input}\PY{o}{=}\PY{l+s+s1}{\PYZsq{}}\PY{l+s+s1}{resources/LEHD\PYZus{}blocks.csv}\PY{l+s+s1}{\PYZsq{}}\PY{p}{,}
                                  \PY{n}{write\PYZus{}to\PYZus{}file}\PY{o}{=}\PY{k+kc}{True}\PY{p}{,}
                                  \PY{n}{load\PYZus{}to\PYZus{}mem}\PY{o}{=}\PY{k+kc}{True}\PY{p}{)}
        
        \PY{c+c1}{\PYZsh{} Run process}
        \PY{n}{w\PYZus{}sym\PYZus{}mat}\PY{o}{.}\PY{n}{process}\PY{p}{(}\PY{p}{)}
        
        \PY{c+c1}{\PYZsh{} Saved as w\PYZus{}sym\PYZus{}mat.csv}
\end{Verbatim}


    \begin{Verbatim}[commandchars=\\\{\}]
{\color{incolor}In [{\color{incolor} }]:} \PY{c+c1}{\PYZsh{} Specify driving distance matrix (takes \PYZti{}30 minutes to run) }
        \PY{n}{d\PYZus{}sym\PYZus{}mat} \PY{o}{=} \PY{n}{TransitMatrix}\PY{p}{(}\PY{n}{network\PYZus{}type}\PY{o}{=}\PY{l+s+s1}{\PYZsq{}}\PY{l+s+s1}{drive}\PY{l+s+s1}{\PYZsq{}}\PY{p}{,}
                                  \PY{n}{primary\PYZus{}input}\PY{o}{=}\PY{l+s+s1}{\PYZsq{}}\PY{l+s+s1}{resources/LEHD\PYZus{}blocks.csv}\PY{l+s+s1}{\PYZsq{}}\PY{p}{,}
                                  \PY{n}{write\PYZus{}to\PYZus{}file}\PY{o}{=}\PY{k+kc}{True}\PY{p}{,}
                                  \PY{n}{load\PYZus{}to\PYZus{}mem}\PY{o}{=}\PY{k+kc}{True}\PY{p}{)}
        
        \PY{c+c1}{\PYZsh{} Run process. For driving, p2p queries OSM to fetch the street network and then output the shortest path transit matrix}
        \PY{n}{d\PYZus{}sym\PYZus{}mat}\PY{o}{.}\PY{n}{process}\PY{p}{(}\PY{n}{speed\PYZus{}limit\PYZus{}filename}\PY{o}{=}\PY{l+s+s1}{\PYZsq{}}\PY{l+s+s1}{resources/condensed\PYZus{}street\PYZus{}data.csv}\PY{l+s+s1}{\PYZsq{}}\PY{p}{)}
        
        \PY{c+c1}{\PYZsh{} Saved as d\PYZus{}sym\PYZus{}mat.csv}
\end{Verbatim}


    Now, snap the points to the units of analysis. However, snapping the
destination points is not always so straightforward. Deciding which
points (laying on the network) are assigned to each area of analysis may
be arbitrary; therefore, it is important to scrutinize the structure of
the data before doing any further processing. If the destinations fall
within the unit of analysis, the best option is to run a within function
that incorporates the destinations to the unit of analysis and then
doing a join with the area IDs. The following image shows that in this
case, we can safely run a function that assigns each point to the area
of analysis of interest.

    

    \textbf{Spatial join of health facilities and area of analysis}

Finally, in order to get the matrix of origins to destinations, we need
to join the health facilities by block with the distance matrix
previously generated. This will generate an asymmetric matrix with all
the distances from destinations to all the units of analysis in Chicago.

    \begin{Verbatim}[commandchars=\\\{\}]
{\color{incolor}In [{\color{incolor} }]:} \PY{c+c1}{\PYZsh{} Read destination files to join with boundaries }
        \PY{n}{health\PYZus{}gdf} \PY{o}{=} \PY{n}{gpd}\PY{o}{.}\PY{n}{read\PYZus{}file}\PY{p}{(}\PY{l+s+s1}{\PYZsq{}}\PY{l+s+s1}{resources/marynia\PYZus{}health.shp}\PY{l+s+s1}{\PYZsq{}}\PY{p}{)}
        \PY{n}{health\PYZus{}gdf}\PY{o}{.}\PY{n}{head}\PY{p}{(}\PY{p}{)}
        \PY{c+c1}{\PYZsh{}Use symmetric matrix calculated above or read your previously saved results:}
        \PY{c+c1}{\PYZsh{}sym\PYZus{}walk=pd.read\PYZus{}csv(\PYZsq{}data/walk\PYZus{}sym.csv\PYZsq{})}
        
        \PY{c+c1}{\PYZsh{} Read boundaries files }
        \PY{n}{boundaries\PYZus{}gdf} \PY{o}{=} \PY{n}{gpd}\PY{o}{.}\PY{n}{read\PYZus{}file}\PY{p}{(}\PY{l+s+s1}{\PYZsq{}}\PY{l+s+s1}{resources/blocks\PYZus{}46265.shp}\PY{l+s+s1}{\PYZsq{}}\PY{p}{)}
        
        \PY{c+c1}{\PYZsh{} Rename the ID name in order to match both data frames. }
        \PY{n}{sym\PYZus{}walk}\PY{o}{=} \PY{n}{sym\PYZus{}walk}\PY{o}{.}\PY{n}{rename}\PY{p}{(}\PY{n}{index}\PY{o}{=}\PY{n+nb}{str}\PY{p}{,} \PY{n}{columns}\PY{o}{=}\PY{p}{\PYZob{}}\PY{l+s+s2}{\PYZdq{}}\PY{l+s+s2}{Unnamed: 0}\PY{l+s+s2}{\PYZdq{}}\PY{p}{:} \PY{l+s+s2}{\PYZdq{}}\PY{l+s+s2}{geoid10}\PY{l+s+s2}{\PYZdq{}}\PY{p}{\PYZcb{}}\PY{p}{)}
        
        \PY{c+c1}{\PYZsh{} Spatial join of amenities within each area of analysis }
        \PY{n}{s\PYZus{}join} \PY{o}{=} \PY{n}{gpd}\PY{o}{.}\PY{n}{sjoin}\PY{p}{(}\PY{n}{health\PYZus{}gdf}\PY{p}{,} \PY{n}{boundaries\PYZus{}gdf}\PY{p}{,} \PY{n}{how}\PY{o}{=}\PY{l+s+s1}{\PYZsq{}}\PY{l+s+s1}{inner}\PY{l+s+s1}{\PYZsq{}}\PY{p}{,} \PY{n}{op}\PY{o}{=}\PY{l+s+s1}{\PYZsq{}}\PY{l+s+s1}{within}\PY{l+s+s1}{\PYZsq{}}\PY{p}{)}
        
        \PY{c+c1}{\PYZsh{} Convert geopanda dataframe to non\PYZhy{}spatial dataframe to join }
        \PY{n}{jb\PYZus{}df} \PY{o}{=} \PY{n}{pd}\PY{o}{.}\PY{n}{DataFrame}\PY{p}{(}\PY{n}{s\PYZus{}join}\PY{p}{)}
        
        \PY{c+c1}{\PYZsh{} Make sure the id is of the same data type in both data frames.}
        \PY{c+c1}{\PYZsh{} sym\PYZus{}walk.dtypes}
        \PY{c+c1}{\PYZsh{} jb\PYZus{}df.dtypes}
        \PY{n}{jb\PYZus{}df}\PY{o}{.}\PY{n}{geoid10}\PY{o}{=}\PY{n}{jb\PYZus{}df}\PY{o}{.}\PY{n}{geoid10}\PY{o}{.}\PY{n}{astype}\PY{p}{(}\PY{n+nb}{int}\PY{p}{)}
        
        \PY{c+c1}{\PYZsh{} Join the symmetric matrix with the spatially joined data (with geoid10 id)}
        \PY{n}{j\PYZus{}asym}\PY{o}{=}\PY{n}{pd}\PY{o}{.}\PY{n}{merge}\PY{p}{(}\PY{n}{sym\PYZus{}walk}\PY{p}{,} \PY{n}{jb\PYZus{}df}\PY{p}{,} \PY{n}{left\PYZus{}on}\PY{o}{=}\PY{l+s+s1}{\PYZsq{}}\PY{l+s+s1}{geoid10}\PY{l+s+s1}{\PYZsq{}}\PY{p}{,} \PY{n}{right\PYZus{}on}\PY{o}{=}\PY{l+s+s1}{\PYZsq{}}\PY{l+s+s1}{geoid10}\PY{l+s+s1}{\PYZsq{}}\PY{p}{,} \PY{n}{how}\PY{o}{=}\PY{l+s+s1}{\PYZsq{}}\PY{l+s+s1}{inner}\PY{l+s+s1}{\PYZsq{}}\PY{p}{)}
\end{Verbatim}


    \begin{Verbatim}[commandchars=\\\{\}]
{\color{incolor}In [{\color{incolor} }]:} \PY{c+c1}{\PYZsh{} Check that the spatial join and join ran appropriately }
        \PY{n}{j\PYZus{}asym}\PY{o}{.}\PY{n}{head}\PY{p}{(}\PY{p}{)}
\end{Verbatim}


    \hypertarget{model-2-asymmetric-matrix}{%
\subsection{\#\# Model 2: Asymmetric
Matrix}\label{model-2-asymmetric-matrix}}

The second model directly creates an asymmetric matrix from destination
points to the centroids of the area of analysis (also takes
\textasciitilde{} 20 min). This approach is most effective when you are
only calculating the distance matrix or a particular distance score
once.

    \begin{Verbatim}[commandchars=\\\{\}]
{\color{incolor}In [{\color{incolor} }]:} \PY{c+c1}{\PYZsh{} Calculate asymmetric distance matrix for walking }
        
        \PY{n}{w\PYZus{}asym\PYZus{}mat} \PY{o}{=} \PY{n}{TransitMatrix}\PY{p}{(}\PY{n}{network\PYZus{}type}\PY{o}{=}\PY{l+s+s1}{\PYZsq{}}\PY{l+s+s1}{walk}\PY{l+s+s1}{\PYZsq{}}\PY{p}{,}
                                   \PY{n}{primary\PYZus{}input}\PY{o}{=}\PY{l+s+s1}{\PYZsq{}}\PY{l+s+s1}{resources/LEHD\PYZus{}blocks.csv}\PY{l+s+s1}{\PYZsq{}}\PY{p}{,}
                                   \PY{n}{secondary\PYZus{}input}\PY{o}{=}\PY{l+s+s1}{\PYZsq{}}\PY{l+s+s1}{resources/marynia\PYZus{}health.csv}\PY{l+s+s1}{\PYZsq{}}\PY{p}{,} 
                                   \PY{n}{write\PYZus{}to\PYZus{}file}\PY{o}{=}\PY{k+kc}{True}\PY{p}{)}
        
        \PY{n}{w\PYZus{}asym\PYZus{}mat}\PY{o}{.}\PY{n}{process}\PY{p}{(}\PY{p}{)}
        
        \PY{c+c1}{\PYZsh{}The output is walk\PYZus{}asym\PYZus{}health.csv (used in the calculation of the access scores)}
\end{Verbatim}


    \begin{Verbatim}[commandchars=\\\{\}]
{\color{incolor}In [{\color{incolor} }]:} \PY{c+c1}{\PYZsh{} Calculate asymmetric distance matrix for driving }
        
        \PY{n}{d\PYZus{}asym\PYZus{}mat} \PY{o}{=} \PY{n}{TransitMatrix}\PY{p}{(}\PY{n}{network\PYZus{}type}\PY{o}{=}\PY{l+s+s1}{\PYZsq{}}\PY{l+s+s1}{drive}\PY{l+s+s1}{\PYZsq{}}\PY{p}{,}
                                   \PY{n}{primary\PYZus{}input}\PY{o}{=}\PY{l+s+s1}{\PYZsq{}}\PY{l+s+s1}{resources/LEHD\PYZus{}blocks.csv}\PY{l+s+s1}{\PYZsq{}}\PY{p}{,}
                                   \PY{n}{secondary\PYZus{}input}\PY{o}{=}\PY{l+s+s1}{\PYZsq{}}\PY{l+s+s1}{resources/nets\PYZus{}all.csv}\PY{l+s+s1}{\PYZsq{}}\PY{p}{,}
                                   \PY{n}{write\PYZus{}to\PYZus{}file}\PY{o}{=}\PY{k+kc}{True}\PY{p}{)}
        
        \PY{n}{d\PYZus{}asym\PYZus{}mat}\PY{o}{.}\PY{n}{process}\PY{p}{(}\PY{n}{speed\PYZus{}limit\PYZus{}filename}\PY{o}{=}\PY{l+s+s1}{\PYZsq{}}\PY{l+s+s1}{resources/condensed\PYZus{}street\PYZus{}data.csv}\PY{l+s+s1}{\PYZsq{}}\PY{p}{)}
        
        \PY{c+c1}{\PYZsh{}The output is drive\PYZus{}asym\PYZus{}health.csv (used in the calculation of the access scores)}
\end{Verbatim}


    \begin{Verbatim}[commandchars=\\\{\}]
{\color{incolor}In [{\color{incolor} }]:} \PY{c+c1}{\PYZsh{} Calculate asymmetric distance matrix for driving (Currently not working)}
        
        \PY{n}{b\PYZus{}asym\PYZus{}mat} \PY{o}{=} \PY{n}{TransitMatrix}\PY{p}{(}\PY{n}{network\PYZus{}type}\PY{o}{=}\PY{l+s+s1}{\PYZsq{}}\PY{l+s+s1}{bike}\PY{l+s+s1}{\PYZsq{}}\PY{p}{,}
                                   \PY{n}{primary\PYZus{}input}\PY{o}{=}\PY{l+s+s1}{\PYZsq{}}\PY{l+s+s1}{resources/LEHD\PYZus{}blocks.csv}\PY{l+s+s1}{\PYZsq{}}\PY{p}{,}
                                   \PY{n}{secondary\PYZus{}input}\PY{o}{=}\PY{l+s+s1}{\PYZsq{}}\PY{l+s+s1}{resources/marynia\PYZus{}health.csv}\PY{l+s+s1}{\PYZsq{}}\PY{p}{,}
                                   \PY{n}{write\PYZus{}to\PYZus{}file}\PY{o}{=}\PY{k+kc}{True}\PY{p}{)}
        
        \PY{n}{b\PYZus{}asym\PYZus{}mat}\PY{o}{.}\PY{n}{process}\PY{p}{(}\PY{p}{)}
        
        \PY{c+c1}{\PYZsh{}The output is bike\PYZus{}asym\PYZus{}health.csv (used in the calculation of the access scores)}
\end{Verbatim}



    % Add a bibliography block to the postdoc
    
    
    
    \end{document}
